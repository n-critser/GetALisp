% Created 2014-03-03 Mon 22:53
\documentclass[11pt]{article}
\usepackage[utf8]{inputenc}
\usepackage[T1]{fontenc}
\usepackage{fixltx2e}
\usepackage{graphicx}
\usepackage{longtable}
\usepackage{float}
\usepackage{wrapfig}
\usepackage{rotating}
\usepackage[normalem]{ulem}
\usepackage{amsmath}
\usepackage{textcomp}
\usepackage{marvosym}
\usepackage{wasysym}
\usepackage{amssymb}
\usepackage{hyperref}
\tolerance=1000
\author{N-CRITSER}
\date{\textit{<2014-02-26 Wed>}}
\title{Slime-on-Eniac.org}
\hypersetup{
  pdfkeywords={},
  pdfsubject={},
  pdfcreator={Emacs 23.4.1 (Org mode 8.2.4)}}
\begin{document}

\maketitle
\tableofcontents



\section{Sign on to eniac}
\label{sec-1}
\begin{verbatim}
$ ssh <user>@eniac.cs.hunter.cuny.edu
$ ssh cslabXX
\end{verbatim}
\section{Get the Slime Repository from github}
\label{sec-2}
\subsection{Clone the git repository}
\label{sec-2-1}
Type this command into your terminal
\begin{verbatim}
$ git clone https://github.com/slime/slime.git
\end{verbatim}

This will create a directory in your home directory
called slime.  Inside is all the slimey goodness. 
\section{Get Emacs to work right}
\label{sec-3}
From your home directory (you landing directory when you sign on)
run this command
\begin{verbatim}
$ more .emacs
\end{verbatim}
\subsection{Nothing happened}
\label{sec-3-1}
if nothing shows up you don't have a .emacs file.  
So lets make one.  

\subsection{Some lisp stuff showed up}
\label{sec-3-2}
That's fine too. We'll add to it. 

\subsection{Within .emacs file you need the code to call slime and ccl64}
\label{sec-3-3}
C-s C-x is emacs speak for Control+s Control+x (save command) 
\begin{verbatim}
$ emacs .emacs
\end{verbatim}
\begin{verbatim}
;; Setup load-path, autoloads and your lisp system
;; Change the path to slime if you cloned somewhere else
(add-to-list 'load-path "~/slime")
(require 'slime-autoloads)
(setq inferior-lisp-program "ccl64")
;; now save C-x C-s  
;; and exit C-x C-c
\end{verbatim}

\section{Test Slime}
\label{sec-4}
\begin{verbatim}
$ mkdir slime-test
$ cd slime-test/
$ emacs  test.lisp
\end{verbatim}


\begin{verbatim}
;; EVERYTHING PAST THIS IS DONE IN EMACS
;; Make a test for slime
(defun add-test (n1 n2)
  (+ n1 n2))
\end{verbatim}
\subsection{Start Slime}
\label{sec-4-1}
\begin{verbatim}
M-x slime <return>
;; slime should start up
;; then split the screen 
C-x 4 b   
;; this gives you a list of buffers
;; usually the first one is the last file you worked on
;; if test.lisp is first hit <RETURN>
;; hit 
C-x o  ;; thats an o not a 0
;; you should be in slime REPL NOW

;; type 
(load "test.lisp") ;; <RETURN>
(add-test 1 2)     ;; <RETURN> 
3
\end{verbatim}
Thats it. Slime is up and running on emacs in eniac. !!!
PS: I tested this on a Chromebook via ssh from the j lab. 
If it doesn't work for you let me know.  
% Emacs 23.4.1 (Org mode 8.2.4)
\end{document}
